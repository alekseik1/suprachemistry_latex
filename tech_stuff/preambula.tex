%%%%%%%%%%%%%%%%%%%%%%%%%%%%%%%%%%%%%%%%%
% Programming/Coding Assignment
% LaTeX Template
%
% This template has been downloaded from:
% http://www.latextemplates.com
%
% Original author:
% Ted Pavlic (http://www.tedpavlic.com)
%
% Note:
% The \lipsum[#] commands throughout this template generate dummy text
% to fill the template out. These commands should all be removed when 
% writing assignment content.
%
% This template uses a Perl script as an example snippet of code, most other
% languages are also usable. Configure them in the "CODE INCLUSION 
% CONFIGURATION" section.
%
%%%%%%%%%%%%%%%%%%%%%%%%%%%%%%%%%%%%%%%%%

%----------------------------------------------------------------------------------------
%	PACKAGES AND OTHER DOCUMENT CONFIGURATIONS
%----------------------------------------------------------------------------------------

\documentclass[a4paper, 12pt, openany]{book}

%%% Работа с русским языком % для pdfLatex
\usepackage{cmap}					% поиск в~PDF
\usepackage{mathtext} 				% русские буквы в~фомулах
\usepackage[T2A]{fontenc}			% кодировка
\usepackage[utf8]{inputenc}			% кодировка исходного текста
\usepackage[english,russian]{babel}	% локализация и переносы
\usepackage{indentfirst} 			% отступ 1 абзаца
%%% ------------------------------------------

\usepackage{fancyhdr} % Required for custom headers
\usepackage{lastpage} % Required to determine the last page for the footer
\usepackage{extramarks} % Required for headers and footers
\usepackage[usenames,dvipsnames]{color} % Required for custom colors
\usepackage{graphicx} % Required to insert images
\usepackage{listings} % Required for insertion of code
\usepackage{courier} % Required for the courier font
\usepackage{lipsum} % Used for inserting dummy 'Lorem ipsum' text into the template
\usepackage[unicode, colorlinks, urlcolor=blue, linkcolor=blue, pagecolor=blue, citecolor=blue]{hyperref}
\usepackage{blindtext}
\usepackage{amsmath}



% Margins
\topmargin=-0.45in
\evensidemargin=0in
\oddsidemargin=0in
\textwidth=6.5in
\textheight=9.0in
\headsep=0.25in


\linespread{1.1} % Line spacing

\def\chaptername{Лекция} % Семинар вместо главы
\setcounter{tocdepth}{1} % Глубина отображения в оглавлении


\makeatletter % Убирает нумерацию на страницах, где \chapter
\renewcommand\chapter{\if@openright\cleardoublepage\else\clearpage\fi
	\thispagestyle{empty}% original style: plain
	\global\@topnum\z@
	\@afterindentfalse
	\secdef\@chapter\@schapter}
\makeatother

% Set up the header and footer
\fancypagestyle{pl}{
	\fancyhead{}
	% НЕ ИСПОЛЬЗУЕТСЯ. Может расширить fancy-линию, но смотрится некрасиво
	%\fancyhfoffset[L]{15mm}% slightly less than 0.25in
	%\fancyhfoffset[RO]{10mm}%
	\renewcommand\headrulewidth{0.4pt} % Size of the header rule
	%\renewcommand\footrulewidth{0.4pt} % Size of the footer rule
	%\lhead{\leftmark} % Top left header
	\fancyhead[CO, CE]{\leftmark} % Top center head
	\fancyhead[R]{\textbf{\thepage}}
	%\fancyfoot[L]{\rightmark} % Top right header
	%\lfoot{\lastxmark} % Bottom left footer. NOT USED
	%\cfoot{} % Bottom center footer
	%\rfoot{\textbf{\thepage}} % Bottom right footer
	\setlength\parindent{10pt} % Removes all indentation from paragraphs
	\fancyfoot{}
}
\pagestyle{pl}

\newcommand\tab[1][1cm]{\hspace*{#1}}

