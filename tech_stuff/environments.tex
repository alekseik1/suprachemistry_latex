% Header and footer for when a page split occurs within a problem environment
\newcommand{\enterProblemHeader}[1]{
	\nobreak\extramarks{#1}{#1 продолжается на следующей странице\ldots}\nobreak
}


% Header and footer for when a page split occurs between problem environments
\newcommand{\exitProblemHeader}[1]{
	\nobreak\extramarks{#1}{#1 продолжается на следующей странице\ldots}\nobreak
	\nobreak\extramarks{#1}{}\nobreak
}



\setcounter{secnumdepth}{1} % Removes default section numbers

\newcounter{lectureCounter}	% Номер лекции
\newcommand{\lectureName}{}


\newenvironment{lecture}[1][Часть \arabic{lectureCounter}]{ % Делает лекцию; имеет дефолтное имя
	\stepcounter{lectureCounter} % Увеличиваем счетчик лекций
	\renewcommand{\lectureName}{#1} % Называем лекцию
	\def\chaptername{Лекция} % Семинар вместо главы
	\chapter{\lectureName} % Помечаем лекцию как главу
	\enterProblemHeader{\lectureName} % Линии сверху и снизу, на пафосе :3
	}{\exitProblemHeader{\lectureName} % Линии сверху и снизу, на пафосе :3
}

\newcommand{\textinbox}[1]{ % Просто текст в рамке
	\noindent\framebox[\columnwidth][c]{
		\begin{minipage}{
				0.98\columnwidth
			}
			#1
		\end{minipage}
	}
}



\newcommand{\lecSectionName}{}
\newcounter{lectureSectionCounter}
\newenvironment{lecSection}[1][Подробнее]{ % Делает секцию лекции
	\stepcounter{lectureSectionCounter}
	\renewcommand{\lecSectionName}{#1} % Называем секцию
	\section{\lecSectionName} % Секция используется как секция
	\enterProblemHeader{\lecSectionName} % Линии сверху и снизу, на пафосе :3
	}{\enterProblemHeader{\lectureName} % Линии сверху и снизу, на пафосе :3
}



\newcommand{\lecSubsectionName}{}
\newcounter{lectureSubsectionCounter}
\newenvironment{lecSubsection}[1]{ % Делает субсекцию секции лекции
	\stepcounter{lectureSubsectionCounter}
	\renewcommand{\lecSubsectionName}{#1} % Называем субсекцию
	\subsection{\lecSubsectionName} % Субсекция используется как субсекция
	\enterProblemHeader{\lectureName\ : \lecSubsectionName} % Линии сверху и снизу, на пафосе :3
}{\enterProblemHeader{\lectureName} % Линии сверху и снизу, на пафосе :3
}

\newtheorem{theorem}{Теорема}
\newtheorem{definition}{Определение}	